\section{Module Testing}
\label{sec:mod_test}

To take advantage of Chisel's emulation capability, we simulate modules of the LUT Tile
and itself as whole. The module level testing is intended to eliminate most of the
bugs. It is relatively easy to provide good coverage as the number of states for the LUT
 Tile is small and many signals are independent of each other. This is because the only elements
with states are FFs inside a BLE. To have good routability, outputs have to be independent 
so each of them can be configured for different signals. Although configuration latches are storage
elements, the signals are treated as regular input vectors to improve module simulation speed. \par

\section{System Testing}
\label{sec:sys_perf}

The system level testing includes two components at the same time: the hardware being
generated, and the tool flow used to generate bitstream. We designed a total of eight
test cases with different focuses. Table~\ref{table:test_cases} lists the test cases
and their testing objective. \par

\begin{table}[htpb]
		\begin{center}
				{\small
				{\tabulinesep=1.2mm
				\begin{tabu}{ | p{1.1in} | p{4.0in} |}    \hline
				Test Case & Description \\ \hline\hline
				single\_inv & A single bit inverter tests essential functions of LUTs and routing resources of the FPGA \\ \hline
				single\_inv\_reg & A single bit inverter with registered output. It adds storage element (FF) testing to single\_inv \\ \hline
				wide\_inv & A wide inverter (32 Bits) tests fundamental functions of LUTs and routing resources of the FPGA \\ \hline
				wide\_inv\_reg & A wide inverter (32 Bits) with registered output. It adds storage element (FF) testing to wide\_inv \\ \hline
				ff\_en & A register (32 Bits) test case for testing if inputs can be correctly stored \\ \hline
				simple\_comp & Simple computations test both computing and storage as the same time \\ \hline
				multi\_consumer & A more evolved test case of simple\_comp, tests multiple sinks routing \\ \hline
				counter & A counter (12 Bits) for testing both intra tile routing and inter tile routing \\ \hline
				\end{tabu}}}
				\caption{System Test Cases
				\label{table:test_cases}}
		\end{center}
\end{table}

To leverage generation capability of the work, we varies different user accessible parameters
of the FPGA and present the parameters in Table~\ref{table:test_params} lists the parameters
we varied. Note some combination of parameters generates a valid FPGA, but the FPGA may be too
small to hold a user design. \par

\begin{table}[htpb]
		\begin{center}
				{\small
				{\tabulinesep=1.2mm
				\begin{tabu}{ | p{0.8in} | p{0.7in} | p{0.7in} | p{0.7in} | p{0.7in} | p{0.9in} | }    \hline
				FPGA Size & LUT Size & CLB Size & IPIN W & CHAN W & CHAN NUM  \\ \hline\hline
				5 by 5 & 6 & 10 & 16 & 16 & 160 \\ \hline
				10 by 10 & 5 & 8 & 12 & 12 & 120 \\ \hline
				25 by 25 & 4 & 6 & 8 & 8 & 80 \\ \hline
				\end{tabu}}}
				\caption{Test Parameters
				\label{table:test_params}}
		\end{center}
\end{table}

It is possible to leverage a seed number feature in VPR for improving testing coverage. 
The placement and routing stage is deterministic based on a seed number provided to VPR. By varying
the seed number, different bitstreams are generated. They should be equivalent circuits. More coverage 
on the hardware can be achieved using the same testing vectors by
placing equivalent circuit bitstreams on the FPGA. \par

In summary, all test cases have passed if the test case fits in the FPGA. If the test case
is too large for the FPGA, VPR reports an error. \par