\section{High Level Reconfiguration Architecture}
\label{sec:reconf_hl}

LUTs, crossbar switches and channel switches are all controlled by
configuration stored in the configuration memory. On a very high level
overview, the configuration memory is similar to an SRAM. However,
there is no row access in reading the memory because all configurations are
used at the same time. When programming the configuration memory, the writing
is done on a row by row basis.  \par

To construct the configuration memory in a logical way and to facilitate
bit generation program, the configuration memory is organized into frames. Each
frame contains the configuration for an entire tile. For example, an IO TILE requires
a frame of configuration while a LUT TILE also requires a frame. Therefore, frames can 
have different sizes, but the width is always 32. This design decision simplifies both 
the FPGA architecture and the bitstream generator. \par

On the top level, frames within a column share the same Config\_In signal. 
Frames within a row share same Config\_En signal bus. Some of the signals in Config\_En 
may not be used in all frames as their size can be different. The unused signals
are synthesized away through optimizations in ASIC flow. 
We decide not to have any state in the configuration circuit so users 
can choose an implementation that fits their need. Fig.~\ref{fig:fpga_reconfig} provides a 
top level view of the configuration frames. \par

\begin{figure}[htp]
	\begin{center}
		\epsfxsize=2.8in
		\epsfbox{fpga_configuration.eps}
		    \renewcommand{\captionfont}{\small}
				\caption{FPGA Configuration
				\label{fig:fpga_reconfig}}
	\end{center}
\end{figure}

\section{Configuration Organization Inside Tiles}
\label{sec:config_tiles}

The configuration memory is a sequence of latches. We used latch in the design to
minimize area overhead as a latch is about 35\% smaller than a flip-flop in a 
65nm process technology we use. In addition, smaller area implies shorter wires
which is instrumental in obtaining high operating frequency. \par

Inside a LUT Tile, five segments of configurations exist logically. The first
one is for LUT configuration. The second is for MUXs inside a BLE. The third
is for XBAR configuration. The fourth is for CB and the fifth is for SB. 
Fig.~\ref{fig:lut_config_org} illustrate the configuration inside a LUT Tile.
An IO Tile has configuration memory for CB and SB only. \par

The work is designed to be expandable, so we also defined configuration format
for MEM Tile and MUL Tile. The first segment controls internal parameters. The rests
 (XBAR, CB, SB) are all common all tiles. An XBAR is not mandatory in MEM Tile and MUL
Tile. \par

\begin{figure}[htpb]
	\begin{center}
		\epsfysize=1.5in 
		\epsfbox{lut_config_org.eps}
		    \renewcommand{\captionfont}{\small}
				\caption{LUT Tile Configuration Organization
				\label{fig:lut_config_org}}
	\end{center}
\end{figure}